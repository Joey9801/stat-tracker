\documentclass{article}
\usepackage[a4paper,left=2cm,top=2cm]{geometry}

\usepackage{parskip}
\usepackage{amsmath}
\usepackage{amssymb}
\usepackage{mathtools}

\begin{document}

The plan:

It's basically Elo; we're using the Gaussian rather than Logistic version for convenience.

In the Elo system, the ``performance'' of players is modelled as a random variable with mean $s_i$ (their skills) and fixed variance:
\[
    X_i \sim N(s_i, \sigma^2)
    \text{.}
\]

Then, the probability that player 1 wins is the probability that their performance exceeds that of the other player
\begin{align*}
    E_1 &= P(X_1 > X_2) = P \left( \underbrace{ \frac{1}{\sigma} \left( X_1 - X_2 - (s_1 - s_2) \right) }_{\mathclap{\sim N(0, 1)}}
                        > - \frac{1}{\sigma} (s_1 - s_2) \right)
         = \Phi\left( \frac{s_2 - s_1}{\sigma} \right) \\
    E_2 &= P(X_2 > X_1) = \Phi\left( \frac{s_1 - s_2}{\sigma} \right) = 1 - E_1
    \text{,}
\end{align*}
and then (supposing player 1 wins) skills are updated:
\begin{align*}
    s_1 &\leftarrow s_1 + k ( 1 - E_A ) \\
    s_2 &\leftarrow s_2 + k ( \underbrace{0 - E_B}_{\mathclap{\text{actual performance} - \text{expected performance}}} )
    \text{,}
\end{align*}
where $k$ is some magic constant. A draw is treated as $\text{actual performance} = \frac{1}{2}$ and therefore a draw between two equally matched people (so $E_A = \frac{1}{2}$) results in no change in scores.

We need to modify this to handle multiple teams, take into account the margin by which each team won. The key points are that:
\begin{itemize}
    \item We can calculate a ``probability of winning something'' from the ``skills''.
    \item The size of the score update is proportional to the probability of the result not occuring.
\end{itemize}

The Setup:
\begin{align*}
    &\text{Teams } && I, J \quad\text{(index sets)} \\
    &\text{Skills } && s_i, s_j \quad (i \in I, j \in J) \\
    &\text{Player performance } && U_i \sim N(s_i, \sigma^2) \\
    &                           && V_i \sim N(s_j, \sigma^2) \\
    &\text{Team performance } && X_I = \frac{1}{|I|} \sum_{i \in I} U_i + K_{|I|} \sim N\left( \frac{1}{|I|} \sum_{i \in I} s_i + K_{|I|}, \quad \frac{1}{|I|} \sigma^2 \right) \\
    &                         && X_J = \frac{1}{|J|} \sum_{i \in J} V_i + K_{|J|} \sim N\left( \frac{1}{|J|} \sum_{j \in J} s_j + K_{|J|}, \quad  \frac{1}{|J|} \sigma^2 \right)
\end{align*}
where $K_{\{1,2,3,4\}}$ is a magic constant.

The magic constant $K$ is to handle the case where the teams are unbalanced.
\begin{align*}
    K_1 = -0.6\sigma &&
    K_2 =  0 &&
    K_3 =  0.2\sigma &&
    K_4 = -0.1\sigma
\end{align*}
If the teams are the same size, the presence of $K$ has no effect on $X_I - X_J$. However, when they are mismatched, it serves to penalise one of the teams.
In words, the above choices assert that given people of equal skill, a three player team is best, followed by two, then four, then one. As an example, those numbers chosen correspond (rougly) to expecting a 10--4 loss when a three player team plays someone on their own.

So, the probability that team $I$ wins any point is:
\[
    E = P( X_I > X_J ) = P\left( \underbrace{ \frac{1}{\varphi} ( X_I - X_J - \tau ) }_{\sim N(0, 1)}
                                 > \frac{-\tau}{\varphi} \right)
                       = \Phi\left( \frac{\tau}{\varphi} \right)
\]
where
\begin{align*}
    \tau &= \frac{1}{|I|} \sum_{i \in I} s_i - \frac{1}{|J|} \sum_{j \in J} s_j + K_{|I|} - K_{|J|} \\
    \varphi &= \sqrt{ \sigma^2 \left( \frac{1}{|I|} + \frac{1}{|J|} \right) }
\end{align*}

We reward team $I$ with $(1 - E)m$ points every time they score (and take the same amount from team $J$; $Em$ points change hands when team $J$ scores (where $m$ is some magic constant).

So if the score were $a$ (team $I$) : $b$ (team J), the score updates would be
\begin{align*}
    s_i &\leftarrow s_i + ((1-E)a - Eb)m && \text{each } i \in I \\
    s_i &\leftarrow s_i + (Eb - (1-E)a)m && \text{each } j \in J
    \text{.}
\end{align*}

Because $(1-E)a - Eb = a - (a + b)E$, you can alternatively, and equivalently, think of this as ``update is proportional to difference between actual, and expected score'' (if score is $a$:$b$, $a + b$ points were played, and we expected $I$ to win $(a+b)E$ of them).

Oh, and as a bit of fun, we can also ``predict'' scores:

If $E > \frac{1}{2}$ we predict a score of
\[
    10 : 10 \frac{1 - E}{E}
\]
and if $E < \frac{1}{2}$
\[
    10 \frac{E}{1 - E} : 10
\]
that is, multiplying $E : (1 - E)$ up so that the larger one is equal to ten.

\end{document}
